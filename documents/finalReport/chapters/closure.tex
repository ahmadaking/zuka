%Hoda Mahmoud 
\section*{Closure}
\bigskip			

Robotic milling is a flexible and efficient method aimed at enhancing and developing the industrial sector. Due to the major advancements in all industrial sectors, milling is targeted for further development to achieve higher accuracy, reduce production costs and losses. This project aimed to achieve four goals; accuracy, ease of use, flexibility and safety in the milling process. These goals were achieved over the course of the project with outstanding results, demonstrated in this report. 
\medskip			
However, the work is still far from over, further developments can be implemented on this project to enhance the obtained results and increase efficiency, level of details obtained, safety and to add further assisting features to the work environment. 
Possible recommendations include using different G-code generating software other than Mesh CAM, as SprutCAM, power mill and Mastercam, as they offer generation in XYZABC dimensions, enabling G-codes in more than 3-axes, thus more axes for the milling process. Safety in the workspace can be increased by several methods including activating the built-in collision detection in the robot. This feature enabled the robot’s links to brake as it detects contact with any external surface. Another feature can be constructing a protective metal cage around the robot to define its workspace and eliminate accidents. In addition to the cage, capacitive sensors can be embedded in the ground around the robot to detect human presence and reduce the speed of the robot or stop it completely.


\medskip Industrial automation is the trend of our age, further research and development work should be targeted at implementing new methods, features and take advantage of the current developments to reach higher targets aimed to increase the prosperity of mankind.