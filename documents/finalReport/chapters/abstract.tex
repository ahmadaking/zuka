\markboth{Abstract}{}
\chapter*{Abstract}\label{ch:abstract}
\addcontentsline{toc}{chapter}{Abstract}
%%\begin{abstract}
%The first industrial revolution marked the transition to new manufacturing processes in eighteenth century, which is arguably similar to the transition introduced by the use of robotic manipulators in different industrial aspects in the late twentieth century. This project is motivated by the major developments in the industrial sector thanks to robots, especially in machining processes. Robots offer more flexibility, cost reduction and higher level of details, all of which are essential characteristics to any successful industry. Although these characteristics can be obtained by conventional CNC machines, however, the level and rate of production differ on larger scales, in favor of the robots. 
%
%The main problem can be summarized in Four points; accuracy, ease of use, flexibility and safety, and the solution to these problems defines the scope of our project. As for accuracy, it is obtained by implementing the robot itself, which offers multiple-axes movement, enabling the possibility for higher level of details than the conventional CNC machines. Ease of use is demonstrated in the user-friendly robot interface that enables the implementation of projects easily and without the need to multiple machines or tasks to deliver the final results. Flexibility is provided through the multiple programmable interfaces that offer multiple methods of control; Inline programming through KUKA’s smartPAD, offline programming through converting G-codes from CAD files into KRL and ROS. Safety is increased in the work space of the robot by introducing a vision based safety system that reduces the robot’s operating speed when someone enters this work space.
%
%The results of the aforementioned methods and applications are diverse, offering milling in multiple dimensions and thus widening the scope of final products. In addition to introducing further control methods, which opens up new doors towards further developments and applications that were not applicable earlier.
%%\end{abstract}
\setlength{\parindent}{1em}
  
This report describes the first commissioning process, and the deployment of both industrial and research applications on KUKA KR6 R900 sixx at Zagazig University. This project is a Bachelor graduation project submitted to the Mechatronics Program on 2017.%, which was purchased through the Tempus project "JIM2L", lead by Hochschule Bochum three years before writing this report.

The project workflow included three major phases: \textbf{Commissioning} the robot- starting from base setting to making complete programs,  \textbf{Implementation of an industrial application}- through developing postprocessing tools to convert conventional G-Code into KUKA Robot Language (KRL) and apply this on milling process, and \textbf{ Implementation of two research applications} related to visual servoing- through using Kinect interfacing on ROS to guide the robot with hand gestures and provide human-safe operating zone.