%Hoda Mahmoud 

\documentclass{report}

\begin{document}	

\clearpage
\thispagestyle{plain}
\par\vspace*{.35\textheight}{\centering In memory of Ahmed Emam\par}


\newpage
	
	
	
\chapter*{Acknowledgements}


We would like to express our deepest appreciation to all those who provided us the possibility to complete this project, most importantly, our deepest gratitude goes to KUKA AG, as without their Robots, our project would have never seen the light .  

A special gratitude goes to our final year project  supervisors, Prof. Ahmed Hamdy and prof. Mohamed Nour Abdelgawad, whose contributions in stimulating suggestions and encouragement,  helped us to coordinate our project, especially in writing this report.
Furthermore I would also like to acknowledge with much appreciation the crucial role of the staff of Zagazig university’s Mechanical Engineering department, who gave the permission to use all required  equipment and the necessary materials to complete our project. We appreciate the guidance given by the supervisors as well as the panels especially in our project presentation that has improved our presentation skills thanks to their comment and advices.

	
	
	\begin{abstract}
	\bigskip			


The first industrial revolution marked the transition to new manufacturing processes in eighteenth century, which is arguably similar to the transition introduced by the use of robotic manipulators in different industrial aspects in the late twentieth century. This project is motivated by the major developments in the industrial sector thanks to robots, especially in machining processes. Robots offer more flexibility, cost reduction and higher level of details, all of which are essential characteristics to any successful industry. Although these characteristics can be obtained by conventional CNC machines, however, the level and rate of production differ on larger scales, in favor of the robots. 
	\medskip			
	
	The main problem can be summarized in Four points; accuracy, ease of use, flexibility and safety, and the solution to these problems defines the scope of our project. As for accuracy, it is obtained by implementing the robot itself, which offers multiple-axes movement, enabling the possibility for higher level of details than the conventional CNC machines. Ease of use is demonstrated in the user-friendly robot interface that enables the implementation of projects easily and without the need to multiple machines or tasks to deliver the final results. Flexibility is provided through the multiple programmable interfaces that offer multiple methods of control; Inline programming through KUKA’s smartPAD, offline programming through converting G-codes from CAD files into KRL and ROS. Safety is increased in the work space of the robot by introducing a vision based safety system that reduces the robot’s operating speed when someone enters this work space.
	
	\medskip			
	
	The results of the aforementioned methods and applications are diverse, offering milling in multiple dimensions and thus widening the scope of final products. In addition to introducing further control methods, which opens up new doors towards further developments and applications that were not applicable earlier.



\end{abstract}
\end{document}