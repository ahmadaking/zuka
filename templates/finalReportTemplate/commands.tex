%!TEX root = myThesis.tex
%!TEX encoding = UTF-8 Unicode

%***************************************************************************************************
% create smaller pdf
% http://tex.stackexchange.com/questions/14429/pdftex-reduce-pdf-size-reduce-image-quality

%  gs -sDEVICE=pdfwrite -dCompatibilityLevel=1.4 -dPDFSETTINGS=/prepress -dNOPAUSE -dQUIET -dBATCH -sOutputFile=small.pdf Doktorarbeit.pdf

%  gs -sDEVICE=pdfwrite -dCompatibilityLevel=1.4 -dPDFSETTINGS=/ebook -dNOPAUSE -dQUIET -dBATCH -sOutputFile=small.pdf Doktorarbeit.pdf

% -dPDFSETTINGS=/screen   (screen-view-only quality, 72 dpi images)
% -dPDFSETTINGS=/ebook    (low quality, 150 dpi images)
% -dPDFSETTINGS=/printer  (high quality, 300 dpi images)
% -dPDFSETTINGS=/prepress (high quality, color preserving, 300 dpi imgs)
% -dPDFSETTINGS=/default  (almost identical to /screen)
%***************************************************************************************************


% settings -------------------------------------------------------------------
%try this to fix the margin problem
%http://tex.stackexchange.com/questions/10128/two-sided-document-reverse-page-margins-for-hardcopy
%***************************************************************************************************

\usepackage{mathtools} %for the dcases environment
%*****************************************************************************
\setlength{\marginparwidth}{0pt}

\newenvironment{myCompactItemize}
{ \begin{itemize}
    \setlength{\itemsep}{0pt}
    \setlength{\parskip}{0pt}
    \setlength{\parsep}{0pt}     }
{ \end{itemize}                  } 

% 'text' shortcuts -----------------------------------------------------------
\newcommand{\etal}{\textit{et al.\ }}
\newcommand{\kmeans}{$k$--{\ttfamily means} }
\newcommand{\kmeanspp}{$k$--{\ttfamily means++} }
\newcommand{\art}{\textit{i}{\scshape ArteC} }

% math stuff -----------------------------------------------------------------
\DeclareMathOperator{\sgn}{sgn}
\DeclareMathOperator*{\argmin}{arg\,min}
\newcommand{\s}[2]{\left\langle #1,#2\right\rangle} % scalar product
\newcommand{\n}[1]{\left\|#1\right\|}  							% norm
\newcommand{\abs}[1]{\left |#1\right |} 						%abs, magnitude


%commented out because it causes  the error:
%too many math alphabets used in version normal
%\usepackage{bm}
%\renewcommand{\vec}[1]{\ensuremath{\bm{#1}}}
%\newcommand{\matx}[1]{\ensuremath{\bm{#1}}}     		%matrix notation (ISO complying version)

\renewcommand{\vec}[1]{\ensuremath{\mathbf{#1}}} 	%vector notation
\newcommand{\matx}[1]{\ensuremath{\mathbf{#1}}} 		% matrix notation


%$\begin{bmatrix*}[r]
  %-1 & 3 \\
  %2 & -4
 %\end{bmatrix*}
%$

% environment redefenitions --------------------------------------------------
\newtheorem{defn}{Method}%{\bfseries}{\itshape}
\theoremstyle{definition} %plain | definition | remark
\newtheorem{definition}{Definition}

%shorthand for the nomenclature that prints the symbol/abbreviation and generates a list entry at the same time.
\newcommand*{\nom}[2]{#1\nomenclature{#1}{#2}}
%example: \nom{EST}{Eastern Standard Time}
%\nom{}{}

%\def\mydate{\leavevmode\hbox{\the\year-\twodigits\month-\twodigits\day}}
\def\mydate{\leavevmode\hbox{\the\year\twodigits\month\twodigits\day}}
\def\twodigits#1{\ifnum#1<10 0\fi\the#1}